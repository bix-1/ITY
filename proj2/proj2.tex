\documentclass[a4paper, 11pt, twocolumn]{article}
\usepackage[left=1.5cm, top=2.5cm, total={18cm, 25cm}]{geometry}
\usepackage[IL2]{fontenc}
\usepackage[utf8]{inputenc}
\usepackage[czech]{babel}
\usepackage{times}
\usepackage{amsmath, amsthm, amssymb}
\newtheorem{definition}{Definice}
\newtheorem{sentence}{Věta}


\begin{document}

\begin{titlepage}

\begin{center}
\Huge
\textsc{Fakulta informačních technologií \\
Vysoké učení technické v Brně} \\
\vspace{\stretch{0.382}}

\LARGE
Typografie a~publikování -- 2.~projekt \\
Sazba dokumentů a~matematických výrazů \\
\vspace{\stretch{0.618}}
\end{center}

{\Large 2021 \hfill
Jakub Bartko (xbartk07)}
\end{titlepage}

\section*{Úvod}
V~této úloze si vyzkoušíme sazbu titulní strany, matematických vzorců, prostředí a~dalších textových struktur obvyklých pro technicky zaměřené texty (například rovnice~(\ref{eq:one}) nebo Definice~\ref{def:one} na straně~\pageref{def:one}). Rovněž si vyzkoušíme používání odkazů \verb|\ref| a~\verb|\pageref|.

Na titulní straně je využito sázení nadpisu podle optického středu s~využitím zlatého řezu. Tento postup byl probírán na přednášce. Dále je použito odřádkování se zadanou relativní velikostí 0.4\,em a~0.3\,em.

V~případě, že budete potřebovat vyjádřit matematickou konstrukci nebo symbol a~nebude se Vám dařit jej nalézt v~samotném \LaTeX u, doporučuji prostudovat možnosti balíku maker \AmS-\LaTeX.

\section{Matematický text}
Nejprve se podíváme na sázení matematických symbolů a~výrazů v~plynulém textu včetně sazby definic a~vět s~využitím balíku \texttt{amsthm}. Rovněž použijeme poznámku pod čarou s~použitím příkazu \verb|\footnote|. Někdy je vhodné použít konstrukci \verb|\mbox{}|, která říká, že text nemá být zalomen.

\begin{definition}\label{def:one}
{\normalfont Rozšířený zásobníkový automat} (RZA) je definován jako sedmice tvaru $A=(Q, \Sigma, \Gamma, \delta, q_0, Z_0, F)$, kde:
\renewcommand\labelitemi{$\bullet$}
\begin{itemize}
\item $Q$ je konečná množina {\normalfont vnitřních (řídicích) stavů},

\item $\Sigma$ je konečná {\normalfont vstupní abeceda},

\item $\Gamma$ je konečná {\normalfont zásobníková abeceda},

\item $\delta$ je {\normalfont přechodová funkce} $Q \times (\Sigma\cup\{\epsilon\})\times\Gamma^\ast\rightarrow2^{Q\times\Gamma^\ast}$,

\item $q_0\in Q$ je {\normalfont počáteční stav}, $Z_0\in\Gamma$ je {\normalfont startovací symbol
zásobníku} a $F \subseteq Q$ je množina {\normalfont koncových stavů}.
\end{itemize}
\end{definition}

Nechť $P=(Q,\Sigma,\Gamma,\delta,q_0,Z_0,F)$ je rozšířený zásobníkový automat. \textit{Konfigurací} nazveme trojici $(q,w,\alpha) \in Q\times\Sigma^\ast\times\Gamma^\ast$, kde $q$~je aktuální stav vnitřního řízení, $w$~je dosud nezpracovaná část vstupního řetězce a~$\alpha=Z_{i_1}Z_{i_2}\dots Z_{i_k}$ je obsah zásobníku\footnote{$Z_{i_1}$
je vrchol zásobníku}.

\subsection{Podsekce obsahující větu a odkaz}
\begin{definition}\label{def:two}
{\normalfont Řetězec $w$ nad abecedou $\Sigma$ je přijat RZA} $A$ jestliže $(q_0,w,Z_0) \overset{*}{\underset{A}\vdash} (q_F,\epsilon,\gamma)$ pro nějaké $\gamma\in\Gamma^\ast$ a $q_F\in F$. Množinu $L(A)=\{w\;|\;w\text{ je přijat RZA }A\} \subseteq\Sigma^\ast$ nazýváme \normalfont{jazyk přijímaný RZA} $A$.
\end{definition}

\newpage
Nyní si vyzkoušíme sazbu vět a~důkazů opět s~použitím balíku \texttt{amsthm}.

\begin{sentence}
Třída jazyků, které jsou přijímány ZA, odpovídá \normalfont{bezkontextovým jazykům}.
\end{sentence}
\begin{proof}
V~důkaze vyjdeme z~Definice~\ref{def:one}~a~\ref{def:two}.
\end{proof}

\section{Rovnice a odkazy}
Složitější matematické formulace sázíme mimo plynulý
text. Lze umístit několik výrazů na jeden řádek, ale pak je třeba tyto vhodně oddělit, například příkazem \verb|\quad|.

\[\sqrt[i]{x^3_i}\quad\text{kde } x_i \text{ je } i\text{-té sudé číslo splňující}\quad x_i^{x_i^{i^2}+2}\leq y_i^{x_i^4}\]

V~rovnici~(\ref{eq:one}) jsou využity tři typy závorek s~různou
explicitně definovanou velikostí.
\begin{eqnarray}
    x &=& \bigg[\Big\{\big[a+b\big]*c\Big\}^d\oplus2\bigg]^{3/2}\label{eq:one} \\
    y &=& \lim_{x\to\infty} \frac{\frac{1}{\log_{10}x}}{\sin^2x + \cos^2x} \nonumber
\end{eqnarray}

V~této větě vidíme, jak vypadá implicitní vysázení limity $\lim_{n\to\infty} f(n)$ v~normálním odstavci textu. Podobně je to i~s~dalšími symboly jako $\prod_{i=1}^n 2^i$ či $\bigcap_{A \in \mathcal{B}}A$. V~případě vzorců $\lim\limits_{n\to\infty} f(n)$ a~$\prod\limits_{i=1}^n 2^i$ jsme si vynutili méně úspornou sazbu příkazem \verb|\limits|.

\begin{eqnarray}
    \int_b^a g(x)\,\mathrm{d}x &=& -\int\limits_a^b f(x)\,\mathrm{d}x
\end{eqnarray}

\section{Matice}
Pro sázení matic se velmi často používá prostředí \texttt{array} a~závorky (\verb|\left|, \verb|\right|).

\[\left( \begin{array}{ccc}
a-b & \widehat{\xi+\omega} & \pi \\
\vec{\mathbf{a}} & \overleftrightarrow{AC} & \hat{\beta}
\end{array} \right) = 1 \Longleftrightarrow \mathcal{Q} = \mathbb{R}\]

\[ \mathbf{A} = \left\|\begin{array}{cccc}
a_{11} & a_{12} & \dots & a_{1n} \\
a_{21} & a_{22} & \dots & a_{2n} \\
\vdots & \vdots & \ddots & \vdots \\
a_{m1} & a_{m2} & \hdots & a_{mn}
\end{array}\right\| = 
\left|\begin{array}{rl}
t & u \\
v & w
\end{array}\right| = tw - uv\]

Prostředí \texttt{array} lze úspěšně využít i~jinde.

\[\binom{n}{k} = \left\{\begin{array}{cl}
0   & \text{pro } k < 0 \text{ nebo } k > n \\
\frac{n!}{k!(n-k)!} & \text{pro } 0 \leq k \leq n.
\end{array}\right.\]

\end{document}