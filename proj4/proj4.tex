\documentclass[a4paper, 11pt]{article}
\usepackage[left=2cm, top=3cm, total={17cm, 24cm}]{geometry}
\usepackage{times}
\usepackage[utf8]{inputenc}
\usepackage[slovak]{babel}
\usepackage[unicode]{hyperref}
\usepackage{multirow}
\usepackage{url}
\DeclareUrlCommand\url{\def\UrlLeft{<}\def\UrlRight{>} \urlstyle{tt}}


\begin{document}

\begin{titlepage}
    \begin{center}
        \Huge
        \textsc{Vysoké učení technické v~Brně} \\
        \huge
        \textsc{Fakulta informačních technologií} \\
        \vspace{\stretch{0.382}}
        \LARGE
        Typografie a~publikování\,--\,4.~projekt \\
        \Huge
        Bibliografické citácie\\
        \vspace{\stretch{0.618}}
    \end{center}
    {\Large
        \today\hfill
        Jakub Bartko
    }
\end{titlepage}

\section{Počiatky typografie}
    Aj keď sa pojem typografia bežne spája s~tlačou a~inými relatívne modernými médiami, podľa jednej z~definícií anglického slova \textit{typography}\,--\,umenie reprezentácie znakmi či symbolmi, alebo umenie tlače~\cite{typography}, môžeme pod tento pojem zaradiť aj prvé ľudské pokusy o~zaznamenanie myšlienok či obrazov. Od odtlačkov rúk na stenách jaskynného systému Cueva de las Manos využívajúcich primitívne farbivá nanášané dutými zvieracími kosťami~\cite{cavepaintings}, až po prvé pečate či mince.

    Vynález systému na tlač s~\uv{pohyblivými}, t.~j.~vymeniteľnými písmenami z~porcelánu sa však datuje až do 10.~storočia v~Číne \cite{chinaMoveable}. Predpokladá sa, že vývoj tlačiarenského lisu v~Európe mohol byť ovplyvnený správami obchodníkov vracajúcich sa z východnej Ázie o~pokrokoch v~produkcií tlačeného slova. Ďalšie informácie o~vplyvoch vynálezov ďalekého východu na európsku kníhtlač nájdete v~\cite{chinaAndEurope}. 
    
\section{Od rukopisu k~prvým lisom}
    V~stredovekej spoločnosti boli prominentné rozsiahle rukopisy, ktoré dali za vznik množstvu rôznorodým štýlom písma, obohatili umenie kaligrafie a~motivovali trendy, ako napr. upustenie od prevažne hranatého písma rukopisov z~čias Ríma, formujúce typografiu do podoby, v~akej ju poznáme dnes~\cite{printmag}.
    
    O~nerozdeliteľné spojenie typografie s~kníhtlačou sa zaslúžil najmä \textbf{Johannes Gutenberg}, ktorého vynález (alebo \uv{prvé zavedenie v~Európe} \cite{gutenberg}) tlačiarenského lisu s~vymeniteľnými písmenami je považovaný za jeden z~najvýznamnejších míľnikov ľudstva, ktorý bol nezameniteľnou súčasťou rozvoja renesancie, reformácie a~osvietenstva. Jeho hlavným prínosom nebol samotný tlačiarenský lis, ktorý využívali už jeho predchodcovia. Jeho najznámejším vynálezom bolo nahradenie blokovej tlače (každá vytlačená strana si vyžadovala samostatný blok z~dreva alebo kovu so statickým obsahom, ktorý bol nevyužiteľný na ďalšiu tlač). Mnohé procesy, praktiky a~ich implementácia do výsledného stroja, ktorých vynájdenie sa mu pripisuje, zaužívaný proces kníhtlače revolucionizovali do bodu, v~ktorom produkcia kníh prestala byť monopolom kláštorov a~majstrovských pisárov. Vynálezy hromadnej výroby vymeniteľného písma, atramentu na báze oleja, dreveného tlačiarenského lisu a~ďalšie tak učinili produkciu a~spotrebu literatúry finančne dostupnou pre tlačiara aj čitateľa (paragraf parafrázovaný z~\cite{gutenberg}).
    
\section{Typografia a~moderná doba}
    Typografia vďaka masovým a~sociálnym médiám prenikla do každodenného života prakticky každého človeka. Zohráva neprehliadnuteľnú rolu vo všetkých formách fyzickej tlače či online dokumentov, od nápisov na spotrebnom tovare, cez reklamy, až po vedecké publikácie. Skúma vplyvy všetkých aspektov písaného slova až do najmenších detailov, ako sú hrúbka písma alebo veľkosť medzier medzi riadkami~\cite{typographyStudy}.
    
    Značná pozornosť sa venuje typografickému aspektu reklám a~ostatných typov prezentácie firiem. Štúdie napríklad ukázali, že teplé alebo čierno-biele farebné kombinácie evokujú väčší pocit dôvery voči produktu a~že vekové skupiny od detí až po starších zákazníkov preferujú bezpätkové fonty písma \cite{sansserif}. Pri každej reklame však dopad zvoleného fontu, veľkosti či farby písma závisí od mnohých okolností, medzi ktoré patrí dĺžka správy, cielená emočná odozva, nálada aj povaha zákazníka, rovnako ako médium, na ktorom sa daná správa nachádza~\cite{adds}.
    
    Pozoruhodným odvetvím typografie sú aj nápisi na budovách, sochách a~pomníkoch, ktoré si prešli vlastnou evolúciou od \uv{tlačeného rímskeho štýlu}, cez fonty \uv{Spacerite} až po dizajn s~využitím počítačových programov a~rezacích strojov \cite{monumentlettering}.

\newpage
\Urlmuskip=0mu plus 0.5mu
\bibliographystyle{czechiso}
\bibliography{proj4}

\end{document}
